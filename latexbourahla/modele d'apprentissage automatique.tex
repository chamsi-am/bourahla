\documentclass[11pt, a4paper]{article}
\usepackage[space]{}

\title{Universite de M'sila \\ Département math et informatique 
\\ Master 2 IA \\Modèle d'apprentissage automatique}
\author{Gherabi Amira}
\date{\today}
\usepackage{forest}
\begin{document}
\maketitle

\section*{exercice n01:}
    \subsection*{a) Les sources du problème d’apprentissage}
    \begin{enumerate}
        \item nombre de cycles durant lesquels la performance 
        de l’agent reste sous-optimale pour la tâche de décision donnée.
        \item Les ressources de calcul nécessaires durant chaque 
        cycle à l’agent pour réviser sa stratégie et choisir une action.



    \end{enumerate}
    \subsection*{b)   Le modèle d’apprentissage:}
    modèle d’apprentissage est un cadre formel donnant une mesure des deux
    sources de complexitè mentionné prècèdament .
    \subsection*{c) L'influence des observations, les actions et le feedback sur la difficulté de l’apprentissage}
    \begin{enumerate}
    \item  L'observation:
          \begin{itemize}
    
              \item[$\ast$] la dimension peut etre immense, voire infini
              \item[$\ast$]  les valeurs de certains attributs peuvent être imprécises, erronées, ou encore absentes.
              \item[$\ast$]  environnements partiellement observables alors ’une situation incertain
            \end{itemize}
     \item  L'action: les actions sont des decision soit simple ou complex, 
    \begin{itemize}
    
        \item[$\ast$] l’espace des décisions possède une structure combinatoire; les décisions peuvent prendre la forme d’arbres, 
        de graphes, ou encore d’hypergraphes
         \item[$\ast$] Les actions simples peuvent avoir un impact sur la difficulté de 
         l’apprentissage selon la manière dont elles influent l’environnement  (épisodique/séquentiel)
      \end{itemize}
      \item  Feedback:Le type de feedback définit le mode d’apprentissage

     \end{enumerate}
     

\section*{exercice n02}

    \subsection*{les composants du problème l’apprentissage de porte logique XOR}
    la fonction Xor envoi une valeur vraie si les deux entrées ne sont
    pas  ́egaux et  fausse si elles sont  ́egaux
    \begin{itemize}
        \item[$\ast$] l’espace des entrées est le couple (a,b)/ (a,b) $ \in $ X=\{(0, 0),(0, 1),(1,0),(1, 1)\} 
\item[$\ast$] l’espace des sorties est Y = \{true,false\}.
\item[$\ast$] programme Xor en python\\
 def xor(x,y):\\
return bool((x and not y) or (not x and y))\\

print(xor(0,0))\\
print(xor(0,1))\\
print(xor(1,0))\\
print(xor(1,1))\\
\end{itemize}
\section*{exercice n03:}
    \subsection*{x_1 \lor x_2 \wedge x_3 \lor x_4}

    \begin{forest}
        [x1, for tree={draw,circle}
           [0] [x2[x3[x4[0][1]][1]][1]]]

        
    \end{forest}
    Liste de décision:

\section*{exercice n04:}
\begin{enumerate}
   \item la différence entre une requète d’appartenance et une
    requète d’equivalence est:
    \begin{enumerate}
        \item[$\ast$] Une requète d’appartenance (MQ) associe à une instance x posée par
        l’apprenant la réponse oui si h(x) = 1, et non sinon.
        \item[$\ast$] Une requète d’ ́equivalence (EQ) associe à une hypothèse h posée par
        l’apprenant la  réponse oui si h = h*, et non sinon
    \end{enumerate}
    \item  Si on a  h1(x1,x2,x3)=x1 ,  h2(x1,x2,x3)=x1  x2 alors l'hypothèse la plus spécifique est h2 
    \item pour toute instance (x1,x2,x3)
    \begin{enumerate}
        \item[$\ast$] Si h(x1,x2,x3) =0 le type de cette requète est requète d’appartenance (MQ) 
        \item[$\ast$] si h*(x1,x2,x3) = 1 requète d’ ́equivalence (EQ)
    \end{enumerate}   
\end{enumerate}


\end{document}

