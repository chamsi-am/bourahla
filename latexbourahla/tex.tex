\documentclass{article}
\usepackage{forest}
\begin{document}

\section*{exercice01hdddddddddhhhhhhhjjjjhhhhh}
    \subsection*{a) les sources du probl`eme d’apprentissage}
    \begin{enumerate}
        \item La premi`ere correspond au nombre de cycles durant lesquels la perfor-
        mance de l’agent reste sous-optimale pour la tˆache de d ́ecision donnee.
        \item La seconde correspond aux ressources de calcul n ́ecessaires durant chaque
        cycle `a l’agent pour r ́eviser sa strat ́egie et choisir une action.



    \end{enumerate}
    \subsection*{b)  une d ́efinition du mod`ele d’apprentissage}
    mod`ele d’apprentissage est un cadre formel donnant une mesure de ces deux
    sources de complexit ́e. Les observations, les actions et le feedback peuvent
    influer sur la difficult ́e de l’apprentissage
    \subsection*{c) Expliquer comment les observations, les actions et le feedback peuvent influer sur la difficult ́e de l’apprentissage}
    not fished


\section*{exercice2}

    \subsection*{Donner avec explication les composants du probl`eme
    l’apprentissage supervis ́e de porte logique XOR}
    Une fonction XOr doit renvoyer une valeur vraie si les deux entr ́ees ne sont
    pas  ́egales et une valeur fausse si elles sont  ́egales -l’espace des entr ́ees est X
    =(0, 1)n et l’espace des sorties est Y = 0,1.
\section*{exercice03}
    \subsection*{x1 x2 x3 x4}

    \begin{forest}
        [x1, for tree={draw,circle}
            [x3[0][x4[0][1]]]
            [x2[x3[0][x4[0][1]]][1]]

        ]
    \end{forest}
\section*{exercice04}
    \subsection*{a) la diff ́erence entre une requˆete d’appartenance et une
    requˆete d’ ́equivalence est}
    \begin{enumerate}
        \item Une requˆete d’appartenance (MQ) associe `a une instance x pos ́ee par
        l’apprenant la r ́eponse oui si h(x) = 1, et non sinon.
        \item Une requˆete d’ ́equivalence (EQ) associe `a une hypoth`ese h pos ́e par
        l’apprenant la r ́eponse oui si h = h, et non sinon
    \end{enumerate}
    \subsection*{b) Si pour toute instance (x1, x2, x3) quelle est l’hypoth`ese
    (le concept) le plus sp ́ecifiqu} est h1((x1,x2,x3)) = x1
    \subsection*{c) Si h(x1,x2,x3) = x1 x2 et la requˆete est h((0,1,1)) =
    0}
    \begin{enumerate}
        \item le type de cette requˆete est requˆete d’appartenance (MQ)
        \item et si h*((0,1,1)) = 1, requˆete d’ ́equivalence (EQ)
        \item Si le contre exemple est (0,1,0) mettre `a jour h. h((0,1,0)) = 1

    \end{enumerate}
\section*{exercice05}
%not solutied
\section*{exercice06}
    \subsection*{la diff ́erence entre le mod`ele agnostique et le mod`ele
    PAC}
    Dans le mod`ele “agnostique”, la distribution D est arbitraire, ce quiimplique
    qu’il n’existe a priori aucune d ́ependance fonctionnelle entre une instance x
    et une d ́ecision y dans un exemple tir ́e dans D. En revanche, dans le mod`ele
    (PAC), nous supposons qu’il existe uned ́ependance fonctionnelle gouvern ́ee
    par une fonction cible

\section*{exercice01}
    \begin{enumerate}
    \item  A[]B = B[]A $\to$ Commutativité
        \item A[]stop = stop[]A = A $\to$ Zéro absorption
    \item  A[](B[]C) = (A[]B)[]C $\to$ Associativité 
        \end{enumerate}
\section*{exercice02}
\section*{exercice03}
    \subsection*{process one [a,b,c] a; (b; stop [] c; stop) endproc}
    \begin{forest}
        [one,for tree={parent anchor=south, child anchor=north, fit=band}
        [a[b[stop]][c[stop]]]]
    \end{forest}
    \subsection*{process two [a,b,c] a; b; stop [] a; c; stop Endproc}
    \begin{forest}
        [two,for tree={parent anchor=south, child anchor=north, fit=band}
        [a[b[stop]]]
        [a[c[stop]]]]
    \end{forest}
    \subsection*{P1 := a; (b; d; stop [] c; stop)}

    \begin{forest}
        [p1,for tree={parent anchor=south, child anchor=north, fit=band}
        [a[b[d[stop]]]
        [c[stop]]]
        ]

    \end{forest}
    
\section*{exercice04}
    \subsection*{Ecrire des spécifications lotos pour les circuits logiques: and, or et 
    xor}
   \subsection*{AND}
    specification circuitlogiqueAND [a,b,c] :noexit \\
    type BIT is \\
        sorts BIT\\
        opns 0 (*! constructor *), \\
            1 (*! constructor *) : -> BIT\\
        and : BIT ,BIT-> BIT\\
        eqns \\
            ofsort BIT \\
            and (0,0) = 0; \\
            and (0,1) = 0; \\
            and (0,1) = 0; \\
            and (1,1) = 1; \\
    endtype\\ 
    behaviour\\ 
        gate_and[a, b, c]\\
    where \\
        process gate_AND[a, b] : noexit :=\\ 
        a ?aa:Bit; b?bb:Bit;  c!and(aa,bb); stop \\
        endproc \\
    endspec\\
    %or
    \subsection*{OR}
    specification circuitlogiqueOR [a,b,c] :noexit \\
    type BIT is \\
        sorts BIT\\
        opns 0 (*! constructor *), \\
            1 (*! constructor *) : -> BIT\\
        or : BIT ,BIT-> BIT\\
        eqns \\
            ofsort BIT \\
            or (0,0) = 0; \\
            or (0,1) = 1; \\
            or (0,1) = 1; \\
            or (1,1) = 1; \\
    endtype\\ 
    behaviour\\ 
        gate_OR[a, b, c]\\
    where \\
        process gate_OR[a, b] : noexit :=\\ 
        a ?aa:Bit; b?bb:Bit;  c!or(aa,bb); stop \\
        endproc \\
    endspec\\
    %xor
    \subsection*{XOR}
    specification circuitlogiqueXOR [a,b,c] :noexit \\
    type BIT is \\
        sorts BIT\\
        opns 0 (*! constructor *), \\
            1 (*! constructor *) : -> BIT\\
        xor : BIT ,BIT-> BIT\\
        eqns \\
            ofsort BIT \\
            and (0,0) = 0; \\
            and (0,1) = 1; \\
            and (0,1) = 1; \\
            and (1,1) = 0; \\
    endtype\\ 
    behaviour\\ 
        gate_XOR[a, b, c]\\
    where \\
        process gate_XOr[a, b] : noexit :=\\ 
        a ?aa:Bit; b?bb:Bit;  c!xor(aa,bb); stop \\
        endproc \\
    endspec\\


\end{document}

